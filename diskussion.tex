\chapter{Diskussion}
Hier erfolgt die Diskussion und Interpretation der Ergebnisse sowie die Beantwortung der Forschungsfrage (ggf. Hypothesenentscheidung) mit:
\begin{itemize}
	\item Einbezug der forschungsmethodologischen Besonderheiten der Untersuchung: z. B. Bezug auf Gütekriterien der Messungen, interne und externe Validität
	\item Bezug zur Literaturanalyse: Diskussion der Ergebnisse vor dem Hintergrund der im Theorieteil dargestellten Grundlagen und problemrelevanten Forschungsergebnisse
	\item Rückschluss auf die Problem- und Fragestellung: problem- und praxisrelevante Folgerungen aus den Ergebnissen (z. B. Folgerungen für die Trainingspraxis in trainingswissenschaftlichen Untersuchungen)
\end{itemize}
Ein weiterer Inhalt kann der sog. \emph{Ausblick} sein, mit:
\begin{itemize}
	\item Verweis auf ungeklärte Probleme,
	\item Wertung der Arbeit in Hinblick auf zukünftige Forschungsansätze sowie
	\item das Aufzeigen von Forschungsperspektiven.
\end{itemize}

