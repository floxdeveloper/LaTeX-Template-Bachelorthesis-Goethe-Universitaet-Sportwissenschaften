\chapter{Problemstellung}
Die Problemstellung beinhaltet die Problemhinführung und Formulierung der allgemeinen Fragestellung, die Beschreibung der Ziele der Arbeit sowie eine Übersicht über die Vorgehensweise und Struktur der Arbeit. Sie kann in \ref{sec:theoretische-grundlagen}~\nameref{sec:theoretische-grundlagen} bzw. \ref{sec:forschungshypothesen}~\nameref{sec:forschungshypothesen} unterteilt werden.
\section{Theoretische Grundlagen}\label{sec:theoretische-grundlagen}
Eine weitere Gliederung könnte \ref{sec:theoretischer-hintergrund}~\nameref{sec:theoretischer-hintergrund} und \ref{sec:forschungsstand}~\nameref{sec:forschungsstand} sein.
\subsection{Theoretischer Hintergrund}\label{sec:theoretischer-hintergrund}
Hier wird bspw. der theoretische Hintergrund aufgearbeitet. Verwendete Blockzitate (mehr als 40 Worte) werden dabei wie folgt formatiert:
Beispielsweise beschreibt Hermann (2001) die Konsequenzen von Verletzungen für Leistungssportler~\footnote{Zur Verbesserung der Lesbarkeit werden in diesen Richtlinien Personenbezeichnungen in der männlichen Form verwendet; gemeint sind dabei in allen Fällen Frauen und Männer.} wie folgt:
\begin{quotation}
	Für Leistungssportler … bedeuten Verletzungen oftmals einen tiefen Einschnitt in den Lebensrhythmus mit unklaren Konsequenzen für die weitere körperliche Leistungsfähigkeit und – damit verbunden – für die weitere sportliche Entwicklung. Je nach individueller Bedeutung des Sports und der Schwere der Läsion können diese Verletzungen mit deutlichen bis massiven psychischen Problemen behaftet sein und für Professionals noch zusätzlich monetär existenzielle Folgen haben. (S. 5)
\end{quotation}


\subsection{Forschungsstand}\label{sec:forschungsstand}
An dieser Stelle wird bspw. der Forschungsstand beschrieben. Eine weitere Gliederung 4. Ebene, wie in den Forschungsstand aktuell (s. Kap.~\ref{sec:aktueller-forschungsstand}) und historisch (s. Kap.~\ref{sec:historischer-forschungsstand}) wäre an dieser Stelle denkbar.
\subsubsection{Aktueller Forschungsstand}\label{sec:aktueller-forschungsstand}
Hier stehen die Ausführungen zum Kap.~\ref{sec:aktueller-forschungsstand}.

\subsubsection{Historischer Forschungsstand}\label{sec:historischer-forschungsstand}
Hier stehen die Ausführungen zum Kap.~\ref{sec:historischer-forschungsstand}.

\section{Forschungshypothesen}\label{sec:forschungshypothesen}
Hier werden die Forschungshypothesen aufgestellt und begründet.