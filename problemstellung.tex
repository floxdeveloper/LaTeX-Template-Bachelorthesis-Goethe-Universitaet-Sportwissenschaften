\chapter{Problemstellung}
Die Problemstellung beinhaltet die Problemhinführung und Formulierung der allgemeinen Fragestellung, die Beschreibung der Ziele der Arbeit sowie eine Übersicht über die Vorgehensweise und Struktur der Arbeit. Sie kann in 1.1 Theoretische Grundlagen bzw. 1.2 Forschungshypothesen unterteilt werden.
\section{Theoretische Grundlagen}
\subsection{Theoretischer Hintergrund}
Hier wird bspw. der theoretische Hintergrund beschrieben.

\subsection{Forschungsstand}
\subsubsection{Aktueller Forschungsstand}
Hier stehen die Ausführungen zum aktuellen Forschungsstand.

\subsubsection{Historischer Forschungsstand}
Hier stehen die Ausführungen zum historischen Forschungsstand.

\section{Forschungshypothesen}
Hier werden die Forschungshypothesen aufgestellt und begründet.